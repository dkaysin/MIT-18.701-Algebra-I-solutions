\documentclass{article}
\usepackage[utf8]{inputenc}
\usepackage[english]{babel}
\usepackage{amsmath}
\usepackage[]{amsthm}
\usepackage[]{amssymb} 
\usepackage{mathrsfs}
\usepackage{tcolorbox}
\usepackage{nicefrac}
\usepackage{mathtools}
% \usepackage{graphicx}
\usepackage{caption}
\usepackage{subcaption}
\usepackage{array}

% \graphicspath{ {./images/} }

\theoremstyle{definition}
\newtheorem*{claim}{Claim}
\newtheorem*{corollary}{Corollary}
\DeclareMathOperator{\adj}{\operatorname{adj}}
\DeclareMathOperator{\im}{\operatorname{im}}
\DeclareMathOperator{\spn}{\operatorname{span}}
\DeclareMathOperator{\nll}{\operatorname{null}}
\newcommand{\trace}{\operatorname{tr}}
\newcommand{\R}{\mathbb{R}}
\newcommand{\Z}{\mathbb{Z}}
\newcommand{\N}{\mathbb{N}}
\newcommand{\F}{\mathbb{F}}
\newcommand{\C}{\mathbb{C}}
\newcommand{\D}{\operatorname{D}}
\newcommand{\GL}{\operatorname{GL}}
\newcommand{\SL}{\operatorname{SL}}
\newcommand{\GLnR}{\GL_n(\R)}
\newcommand{\SLnR}{\SL_n(\R)}
\DeclarePairedDelimiter\floor{\lfloor}{\rfloor}
\DeclarePairedDelimiter\set{\{}{\}}
\DeclarePairedDelimiter\abs{\lvert}{\rvert}
\DeclarePairedDelimiter\genby{\langle}{\rangle}
\newcommand{\restrict}[1]{ \big|_{#1} }


\title{18.701: Problem Set 8}
\author{Dmitry Kaysin}
\date{June 2020}
\begin{document}
\maketitle 


\subsection*{Problem 1}

\begin{tcolorbox}
Let $G$ be a group of order $55$.

a) Prove that $G$ is generated by two elements $x$ and $y$, with the relations $x^{11}=1, \> y^5=1, \> yxy^{-1}=x^r, \text{ for some $r: \; 1 \leq r < 11$}$.

b) Decide which values of $r$ are possible.

c) Prove that there are two isomorphism classes of groups of order $55$.
\end{tcolorbox}

\begin{proof}

By the First Sylow theorem, group $G$ of order $55$ contains at least one Sylow $11$-subgroup $H_{11}$ and at least one Sylow $5$-subgroup $H_5$.

By the Third Sylow theorem, the number of Sylow $11$-subgroups in $G$, must divide $5$ and also must be congruent to $1$ modulo $11$.
Therefore, there is only one $11$-subgroup in $G$, and it must be normal, denote it $H_{11}$.

Since $H_{11}$ is normal, by the First Isomorphism theorem, $G/H_{11}$ is isomorphic to a subgroup of order $55/11=5$, one of the Sylow $5$-subgroups, denote it $H_{5}$.

Since $H_{11}$ and $H_{5}$ have prime order, they are both cyclic, abelian, and they are generated by any of their respective elements other than identity:
\begin{gather*}
    H_{11} = \genby{x}, \> x \neq 1, \> x^{11}=1, \\
    H_{5} = \genby{y}, \> y \neq 1, \> y^{5}=1.
\end{gather*}

Since cosets of $H_{11}$ partition $G$ and $G / H_{11}$ is isomorphic to $H_{5}$, any element of $G$ can be represented as a product of $x^p y^q$ for some $0 \leq p < 11, \> 0 \leq q < 5$.
Therefore, $x$ and $y$ generate $G$. and $H_{11} H_{5} = G$.

We note that since $H_{11}$ is normal, conjugate of $x \in H_{11}$ must be in $H_{11}$, i.e. for $x \neq 1$:
\[
    yxy^{-1} = x^r, \> 1 \leq r < 11.
\]

By the Third Sylow theorem, the number of $5$-subgroup in $G$, $s$, must divide $11$ and must be congruent to $1$ modulo $5$.
There are two such options: $s=1$ and $s=11$, which correspond to two possible isomorphism classes of groups of order $55$.

\paragraph{Case 1. There is only one $5$-subgroup in $G$, namely $H_{5}$.}

Since both $H_{11}$ and $H_{5}$ are abelian, $yx = xy$ and $yxy^{-1} = x$.
Therefore, $r = 1$ for the case $s=1$.

Since there is only one $5$-subgroup of $G$, it must be normal.
We can also see that $H_{11} \cap H_{5} = 1$.
Thus, multiplication map $f:H_{11} \times H_{5} \to G$, defined as $f(h,k) = hk$, is an isomorphism.
We conclude that $G$ is isomorphic to $H_{11} \times H_{5}$ for the case $s=1$.

\paragraph{Case 2. There are 11 $5$-subgroups in $G$.}

Since $x y^5 = y^5 x = 1$, we have:
\begin{gather*}
    x = y^5 x y^{-5} = y^4 (yxy^{-1}) y^{-4} = \\
    \intertext{since $yxy^{-1}=x^r$:}
    = y^4 x^r y^4 = y^3 (y x^r y^{-1}) y^{-3} =
    \intertext{since $(yxy^{-1})^r = y x^r y^{-1}$:}
    = y^3 (y x y^{-1})^r y^{-3} = y^3 (x^r)^r y^{-3} =
    \intertext{continuing:}
    = y^2 x^{(r^3)} y^2 = y x^{(r^4)} y = x^{(r^5)}.
\end{gather*}
Therefore, $r^5$ must be congruent to $1$ modulo order of $x$:
\[ 
    r^5 = 1 \mod 11.
\]
We test possible integer $r$, such that $1 < r < 11$:
\begin{align*}
    2^5 = 32 = 10 & \mod 11, \\
    3^5 = 243 = 1 & \mod 11, \\
    4^5 = 1024 = 1 & \mod 11, \\
    5^5 = 3125 = 1 & \mod 11, \\
    6^5 = 7776 = 10 & \mod 11, \\
    7^5 = 16807 = 10 & \mod 11, \\
    8^5 = 32768 = 10 & \mod 11, \\
    9^5 = 59049 = 1 & \mod 11, \\
    10^5 = 100000 = 10 & \mod 11.
\end{align*}
Therefore, possible values of $r$ for case of $s=11$ are $3,4,5$ and $9$.

We will prove that groups $G_r$ generated by
 \[
    \genby{x, y; x^{11}=1, y^{5}=1, yxy^{-1}=x^r}
 \]
 are isomorphic for $r \in \set{ 3,4,5,9 }$.

Consider group $G_3$ that has is generated by the following relation:
\[
    y x y^{-1} = x^3.
\]
Also consider element $a = y^2$ of the subgroup $H_5$ of $G_3$:
\[
   a x a^{-1} = y^2 x y^{-2} = y (yxy^{-1}) y^{-1} = y x^3 y^{-1} = (x^3)^3 = x^9.
\]
We note that $a$ has order $5$ and generates $H_{5}$.
Thus, substituting $a$ for $y$ and keeping other relations unchanged generates $G_9$.
Therefore, $G_3$ is isomorphic to $G_9$.

By the same logic, for $r=4$ we substitute $b = y^3$ for $y$ and we have:
\[
    b x b^{-1} = y^3 x y^{-3} = (x^4)^3 = (x^{11})^5 x^9 = x^9.
\]
For $r=5$ we substitute $c = y^4$ for $y$ and we have:
\[
    c x c^{-1} = y^4 x y^{-4} = (x^5)^4 = (x^{11})^{56} x^9 = x^9.
\]
We conclude that $G_3 \simeq G_4 \simeq G_5 \simeq G_9$, which constitutes an isomorphism class for the case $s=11$.

\end{proof}


\subsection*{Problem 2}

\begin{tcolorbox}
Use the Todd-Coxeter Algorithm to determine the order of the group generated by two elements $x, y$.

a) with relations $x^3 = 1, \> y^2 = 1, \> yxyxy = 1$.
\end{tcolorbox}

We select $H = \genby{x}$ and denote it as $1$.
First steps of the Todd-Coxeter Algorithm are as follows:

\begin{center}
    \begin{tabular}{l l l l} 
        \hline
        $\quad \> x$ & $\quad \> x$ & $\quad \> x$ \\
        \hline
        $1$ & $1$ & $1$ & $1$ \\ 
        $2$ & $3$ & $4$ & $2$ \\ 
        $3$ & $4$ & $2$ & $3$ 
        \\\hline
    \end{tabular}
\end{center}

\begin{center}
    \begin{tabular}{l l l} 
        \hline
        $\quad \> y$ & $\quad \> y$\\
        \hline
        $1$ & $2$ & $1$ \\ 
        $3$ & $4$ & $3$
        \\\hline
    \end{tabular}
\end{center}

\begin{center}
    \begin{tabular}{l l l l l l} 
        \hline
        $\quad \> y$ & $\quad \> x$ & $\quad \> y$ & $\quad \> x$ & $\quad \> y$ \\
        \hline
        $1$ & $2$ & $3$ & $4$ & $2$ & $1$ \\ 
        $2$ & $1$ & $1$ & $2$ & $3$ & $4$
        \\\hline
    \end{tabular}
\end{center}

At this point we can see that coset $2$ is the same as coset $4$, which implies that cosets $2$ and $3$ are the same.
This immediately "collapses" the group to a trivial group.

\begin{tcolorbox}
b) with relations $x^3 = 1, \> y^4 = 1, \> xyxy = 1$.
\end{tcolorbox}

We select $H = \genby{x}$ and denote it as $1$.
Full table after applying the Todd-Coxeter Algorithm is as follows:

\begin{center}
    \begin{tabular}{l l l l} 
        \hline
        $\quad \> x$ & $\quad \> x$ & $\quad \> x$ \\
        \hline
        $1$ & $1$ & $1$ & $1$ \\ 
        $2$ & $3$ & $4$ & $2$ \\ 
        $5$ & $6$ & $7$ & $5$ \\
        $8$ & $8$ & $8$ & $8$
        \\\hline
    \end{tabular}
\end{center}

\begin{center}
    \begin{tabular}{l l l l l} 
        \hline
        $\quad \> y$ & $\quad \> y$ & $\quad \> y$ & $\quad \> y$ \\
        \hline
        $1$ & $2$ & $6$ & $3$ & $1$ \\ 
        $4$ & $5$ & $8$ & $7$ & $4$
        \\\hline
    \end{tabular}
\end{center}

\begin{center}
    \begin{tabular}{l l l l l} 
        \hline
        $\quad \> x$ & $\quad \> y$ & $\quad \> x$ & $\quad \> y$ \\
        \hline
        $1$ & $1$ & $2$ & $3$ & $1$ \\ 
        $2$ & $3$ & $1$ & $1$ & $2$ \\ 
        $3$ & $4$ & $5$ & $6$ & $3$ \\ 
        $4$ & $2$ & $6$ & $7$ & $4$ \\ 
        $5$ & $6$ & $3$ & $4$ & $5$ \\ 
        $6$ & $7$ & $4$ & $2$ & $1$ \\ 
        $7$ & $5$ & $8$ & $8$ & $7$ 
        \\\hline
    \end{tabular}
\end{center}

Therefore, permutation representations are as follows:
\[
    x = (\,2\,3\,4\,)(\,5\,6\,7\,), \quad
    y = (\,1\,2\,6\,3\,)\,(\,4\,5\,8\,7\,).
\]
Order of $\genby{x}$ is $3$ and the number of cosets of $\genby{x}$ is $8$.
Therefore, the order of the group is $24 = 3 \cdot 8$.

\subsection*{Problem 3}

\begin{tcolorbox}
Classify groups that are generated by two elements $x$ and $y$ of order $2$.

Hint: It will be convenient to make use of the element $z = xy$.
\end{tcolorbox}

Consider group $G$, which is generated by $x$ and $y$ such that $x^2 = y^2 = 1$.
Denote element $z = xy$.

We first notice that $G$ is generated by $\set{z,y}$ since $zy = xyy = x$.
We also notice that:
\[
    zyzy = xyyxyy = x^2 = 1.
\]

Presentation for $G$ can be written as:
\begin{equation} \label{infdih}
    G = \genby{z, y \> | \> y^2 = zyzy = 1}.
\end{equation}
It is easy to see that subgroup $\genby{z}$ of $G$ has infinite order, thus $G$ is infinite.
Presentation \ref{infdih} is the usual presentation for infinite dihedral group.

In case $G$ is finite, $z$ must have finite order; denote it $n$.
Presentation of $G$ is then:
\begin{equation} \label{dihedral}
    G = \genby{z, y \> | \> z^n = y^2 = zyzy = 1}.
\end{equation}
We claim that $G$ is isomorphic to dihedral group $D_{2n}$.

\begin{proof}

Dihedral group $D_{2n}$ is generated by two elements.
Thus, by Artin, Corollary 7.10.14, there exists a surjective homomorphism $\varphi : \mathcal{G} \to G$.
We need to prove that $\varphi$ is injective.
To do that we will show that the order of $G$ is $2n$ using the Todd-Coxeter algorithm. We compute operations of elements of $G$ on the cosets of $\genby{z}$ where $\genby{z}$ is represented as $1$ in the table below:

\begin{center}
    \begin{tabular}{l l l l l} 
        \hline
        $\quad \> z$ & $\quad \> z$ & $\quad \> \cdots$ & $\quad \> z$ \\
        \hline
        $1$ & $1$ & $\cdots$ & $1$ & $1$ \\ 
        $2$ & $2$ & $\cdots$ & $2$ & $2$
        \\\hline
    \end{tabular}
\end{center}

\begin{center}
    \begin{tabular}{l l l} 
        \hline
        $\quad \> y$ & $\quad \> y$ \\
        \hline
        $1$ & $2$ & $1$ \\ 
        $2$ & $1$ & $2$
        \\\hline
    \end{tabular}
\end{center}

\begin{center}
    \begin{tabular}{l l l l l} 
        \hline
        $\quad \> z$ & $\quad \> y$ & $\quad \> z$ & $\quad \> y$ \\
        \hline
        $1$ & $1$ & $2$ & $2$ & $1$ \\ 
        $2$ & $2$ & $1$ & $1$ & $2$
        \\\hline
    \end{tabular}
\end{center}
We conclude that the index of $\genby{z}$ is $2$.
Since the order of $\genby{z}$ is $n$, the order of $G$ must be $2n$, which implies that $\varphi$ is injective.

Therefore, $G$ is isomorphic to $D_{2n}$.

\end{proof}

Introducing additional relations between $z$ and $y$ will result in more elements of $G$ collapsing to $1$.

\end{document}
