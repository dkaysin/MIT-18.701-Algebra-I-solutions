\documentclass{article}
\usepackage[utf8]{inputenc}
\usepackage[english]{babel}
\usepackage{amsmath}
\usepackage[]{amsthm}
\usepackage[]{amssymb} 
\usepackage{mathrsfs}
\usepackage{tcolorbox}
\usepackage{nicefrac}
\usepackage{mathtools}
% \usepackage{graphicx}
\usepackage{caption}
\usepackage{subcaption}
\usepackage{array}

% \graphicspath{ {./images/} }

\theoremstyle{definition}
\newtheorem*{claim}{Claim}
\newtheorem*{corollary}{Corollary}
\DeclareMathOperator{\adj}{\operatorname{adj}}
\DeclareMathOperator{\im}{\operatorname{im}}
\DeclareMathOperator{\spn}{\operatorname{span}}
\DeclareMathOperator{\nll}{\operatorname{null}}
\newcommand{\trace}{\operatorname{trace}}
\newcommand{\R}{\mathbb{R}}
\newcommand{\Z}{\mathbb{Z}}
\newcommand{\N}{\mathbb{N}}
\newcommand{\F}{\mathbb{F}}
\newcommand{\C}{\mathbb{C}}
\newcommand{\D}{\operatorname{D}}
\newcommand{\GL}{\operatorname{GL}}
\newcommand{\SL}{\operatorname{SL}}
\newcommand{\GLnR}{\GL_n(\R)}
\newcommand{\SLnR}{\SL_n(\R)}
\DeclarePairedDelimiter\floor{\lfloor}{\rfloor}
\DeclarePairedDelimiter\set{\{}{\}}
\DeclarePairedDelimiter\abs{\lvert}{\rvert}
\DeclarePairedDelimiter\genby{\langle}{\rangle}
\DeclarePairedDelimiter\bilform{\langle}{\rangle}
\newcommand{\restrict}[1]{ \big|_{#1} }
\newcommand{\evalat}[2]{\Big|_{#1}^{#2}}


\title{18.701: Problem Set 9}
\author{Dmitry Kaysin}
\date{July 2020}
\begin{document}
\maketitle 


\subsection*{Problem 1}

\begin{tcolorbox}
Find an orthonormal basis for the vector space $P_2$ of all real polynomials of degree at most $2$ with the symmetric form defined by 
\[
    \bilform{f,g} = \int_{-1}^1 f(x) g(x) dx .
\]
\end{tcolorbox}

\begin{proof}

We first check that the form $\bilform{\cdot,\cdot}$ is indeed symmetric.
This is the case since integration is linear and multiplication of real polynomials is commutative.

To find an orthonormal basis with the given symmetric form we will apply the Gram-Schmidt procedure.
We start with an arbitrary basis of $P_2$:
\[
    B = \set{1, x, x^2}.
\]
We will determine orthogonal basis $B'$ of $P_2$ first.

We start with the first vector $v_1 = 1$ of $B$.
This will be the first vector of $B'$.

We notice that
\[
    \bilform{1,x} = \int^1_{-1} x \, dx = \frac{x^2}{2} \evalat{-1}{1} = 0,
\]
which means that $v_2 = x$ is orthogonal to $v_1 = w_1 = 1$.
We take $w_2 = x$ as the second vector of $B'$.

We proceed by finding the orthogonal projection of the third vector $v_3 = x^2$ onto the subspace spanned by the orthogonal vectors $\set{w_1, w_2}$:
\[
    \pi(v_3) = w_1 \frac{\bilform{w_1, v_3}}{\bilform{w_1,w_1}} + w_2 \frac{\bilform{w_2, v_3}}{\bilform{w_2,w_2}},
\]
where
\begin{align*}
    \bilform{w_1, v_3} & = \int^1_{-1} x^2 \, dx 
    = \frac{x^3}{3} \evalat{-1}{1} = \frac{2}{3}, \>
    && \bilform{w_1, w_1} 
        = \int^1_{-1} dx = x \evalat{-1}{1} = 2, \\
    \bilform{w_2, v_3} & = \int^1_{-1} x^3 \, dx
    = \frac{x^4}{4} \evalat{-1}{1} = 0,
    && \bilform{w_2, w_2} 
        = \int^1_{-1} x^2 \, dx = \frac{x^3}{3} \evalat{-1}{1} = \frac{2}{3}.
\end{align*}
Thus
\[
    \pi(v_3) = w_1 \frac{\bilform{w_1, v_3}}{\bilform{w_1,w_1}} + w_2 \frac{\bilform{w_2, v_3}}{\bilform{w_2,w_2}}
    = \frac{1}{3}.
\]
Since $v_3 - \pi(v_3)$ is orthogonal to both $w_1$ and $w_2$, we take it as the third vector of $B'$:
\[
    w_3 = x^2 - \frac{1}{3}.
\]

We now have an orthogonal basis $B' = \set{1,x,x^2 - \frac{1}{3}}$.
We normalize each vector in $B'$:
\begin{align*}
    w'_1 & = \frac{w_1}{\bilform{w_1, w_1}^{1/2}} = \frac{\sqrt{2}}{2}, \\
    w'_2 & = \frac{w_2}{\bilform{w_2, w_2}^{1/2}} = \sqrt{\frac{3}{2}} x, \\
    w'_3 & = \frac{w_3}{\bilform{w_3, w_3}^{1/2}} = \frac{3\sqrt{5}}{2\sqrt{2}} \left( x^2 - \frac{1}{3} \right),
\end{align*}
where
\begin{align*}
    \bilform{w_3, w_3} 
        & = \bilform*{x^2-\frac{1}{3}, x^2-\frac{1}{3}}
        = \int^1_{-1} \left( x^2-\frac{1}{3} \right)^2 \, dx 
        = \int^1_{-1} \left( x^4 - \frac{2}{3} x^2 + \frac{1}{9} \right) dx \\
    & = \left( \frac{x^5}{5} - \frac{2}{9}x^3 + \frac{1}{9}x \right) \evalat{-1}{1}
    = \frac{2}{5} - \frac{4}{9} + \frac{2}{9} = \frac{8}{45}.
\end{align*}

Thus, $\set*{
    \frac{\sqrt{2}}{2}, \>
    \sqrt{\frac{3}{2}} x, \>
    \frac{3\sqrt{5}}{2\sqrt{2}} \left( x^2 - \frac{1}{3} \right)
}$ is an orthonormal basis of $P_2$.

\end{proof}


\subsection*{Problem 2}

\begin{tcolorbox}
Let $V$ be the real vector space of $3 \times 3$ matrices with the bilinear form $\bilform{A,B} = \trace A^t B$,
and let $W$ be the subspace of skew-symmetric matrices.
Compute the orthogonal projection to $W$ with respect to this form, of the matrix
\[
    v = 
    \begin{pmatrix}
        1 & 2 & 0 \\
        0 & 0 & 1 \\
        1 & 3 & 0
    \end{pmatrix}.
\]
\end{tcolorbox}

Real $3 \times 3$ skew-symmetric matrices have the form
\[ 
    \begin{pmatrix}
        0 & a & b \\
        -a & 0 & c \\
        -b & -c & 0
    \end{pmatrix}.
\]
Thus vector space $W$ has dimension $3$ with one basis given as the set of vectors
\[
    w_1 = \begin{pmatrix}
        0 & 1 & 0 \\
        -1 & 0 & 0 \\
        0 & 0 & 0
    \end{pmatrix},
    \>\>
    w_2 = \begin{pmatrix}
        0 & 0 & 1 \\
        0 & 0 & 0 \\
        -1 & 0 & 0
    \end{pmatrix},
    \>\>
    w_3 = \begin{pmatrix}
        0 & 0 & 0 \\
        0 & 0 & 1 \\
        0 & -1 & 0
    \end{pmatrix}.
\]
One can easily verify by direct computation that all three vectors $w_1, w_2, w_3$ are orthogonal and
$\bilform{w_1,w_1} = \bilform{w_2,w_2} = \bilform{w_3,w_3}$ = 2.

Therefore, orthogonal projection of the matrix $v$ to the subspace $W$ can be calculated as follows:
\begin{align*}
    \pi_W(v) 
    & = \frac{\bilform{w_1, v}}{\bilform{w_1, w_1}} w_1
    \frac{\bilform{w_2, v}}{\bilform{w_2, w_2}} w_2
    \frac{\bilform{w_2, v}}{\bilform{w_2, w_2}} w_3 \\
    & = \frac{2}{2} w_1 + \frac{-1}{2} w_2 + \frac{-2}{2} w_3
    = w_1 - \frac{1}{2} w_2 - w_3,
\end{align*}
which corresponds to the matrix
\[
    \begin{pmatrix}
        0 & 1 & -\frac{1}{2} \\
        -1 & 0 & -1 \\
        \frac{1}{2} & 1 & 0
    \end{pmatrix}.
\]

\subsection*{Problem 3}

\begin{tcolorbox}
Let $W$ be a two-dimensional subspace of $\R^3$, and consider the orthogonal projection $\pi$ of $\R^3$ onto $W$.
Let $(a_i, b_i)^t$ be the coordinate vector of $\pi(e_i)$, with respect to a chosen orthonormal basis of $W$.
Prove that $(a_1, a_2, a_3)$ and $(b_1, b_2, b_3)$ are orthogonal unit vectors.
\end{tcolorbox}

\begin{proof}

Let $B = \set{ u, v }$ be the chosen orthonormal basis of $W$ and suppose, $u$ has coordinates $(u_1, u_2, u_3)^t$ and $w$ has coordinates $(w_1,w_2,w_3)^t$ in the standard basis.

Applying the projection formula:
\[
    \pi(e_1) =
    \bilform{u, e_1} u + \bilform{v, e_1} v 
    = \bilform*{
        \begin{pmatrix}
            u_1 \\ u_2 \\ u_3
        \end{pmatrix},
        \begin{pmatrix}
            1 \\ 0 \\ 0
        \end{pmatrix}
    } u
    +
    \bilform*{
        \begin{pmatrix}
            w_1 \\ w_2 \\ w_3
        \end{pmatrix},
        \begin{pmatrix}
            1 \\ 0 \\ 0
        \end{pmatrix}
    } w
    = u_1 u + w_1 w.
\]
Therefore, $a_1 = u_1, \> b_1 = w_1$.

Similarly we find coordinates for projections of other standard basis vectors and combine them into row vectors
\begin{align*}
    (a_1, a_2, a_3) = (u_1, u_2, u_3) & = u^t, \\ 
    (b_1, b_2, b_3) = (w_1, w_2, w_3) & = w^t.
\end{align*}
Since $u$ and $w$ are orthogonal, $u^t$ and $w^t$ are also orthogonal.

\end{proof}


\subsection*{Problem 4}

\begin{tcolorbox}
Prove that the signature of the form does not depend on the choice of the orthogonal basis (Sylvester's Law).
\end{tcolorbox}

We first prove that if $W^+$ and $W^-$ are subspaces of $V$ and if the form is positive definite on $W^+$ and negative definite on $W^-$, then $W^+$ and $W^-$ are independent.

\begin{proof}

It is suffices to show that $W^+ \cap W^- = 0$.
Consider vector $w \in W^+ \cap W^-$.
Then both
\[
    \bilform{w,w} \leq 0
    \> \text{ and } \>
    \bilform{w,w} \geq 0
\]
must be true.
Since the form is nondegenerate on $V$, we conclude that $w = 0$.

\end{proof}

Let $V$ be complex vector space equipped with a Hermitian form that is nondegenerate on $V$.
Let $B = \set{ v_1, \dots, v_k, v_{k+1}, \dots, v_{k+n} }$ be an orthonormal basis of $V$ such that
\[
    I_{k, n} =
    \begin{pmatrix}
        I_k & \\
        & -I_n
    \end{pmatrix}
\]
is the matrix of the form with respect to basis $B$.
Denote $W^+ = \spn \set{v_1, \dots, v_k}$, and denote $W^- = \spn \set{v_{k+1}, \dots, v_{k+n}}$.
Choose $B' = \set{ w_1, \dots w_{k+n} }$, another orthonormal basis of $V$ such that matrix of the form with respect to basis $B'$ is diagonal with entries $1, -1$.
We will prove that matrix of the form with respect to $B'$ is the same as the matrix of the form with respect to basis $B$.

\begin{proof}

Ever vector of $B'$ is either an element of $W^+$ or $W^-$.
There are exactly $k$ vectors of $B'$ that are elements of $W^+$ since $W^+$ is of dimension $k$ and vectors of $B'$ are independent.
Similarly, there are exactly $n$ vectors of $B'$ that are elements of $W^-$.
If $w_i$ is in $W^+$, then $\bilform{w_i, w_i} = 1$ since $B'$ is orthonormal.
Conversely, if $w_i$ is in $W^-$, then $\bilform{w_i, w_i} = -1$.
Therefore, matrix of the form with respect to basis $B'$ (after rearranging basis vector appropriately) is as follows: 
\[
    I_{k, n} =
    \begin{pmatrix}
        I_k & \\
        & -I_n
    \end{pmatrix}.
\]

\end{proof}


\subsection*{Problem 5}

\begin{tcolorbox}
Let $T$ be a linear operator on $V = \R^n$ whose matrix $A$ is a real symmetric matrix.

a) Prove that $V$ is the orthogonal sum $V = (\im T) \oplus (\ker T)$.
\end{tcolorbox}

\begin{proof}

We assume the form defined on $V$ to be the standard dot product, which is positive definite.
Since $A$ is a real symmetric matrix, then by the Spectral theorem, there exists an orthonormal basis of $V$ consisting of eigenvectors of $A$, denote it $B = \set{v_1, v_2, \dots, v_n}$.
Consider image of each basis vector of $B$ under $T$:
\[
	\begin{pmatrix}
		v_1 \\
		v_2 \\
		\vdots \\
		v_n
	\end{pmatrix}
	\longmapsto
	\begin{pmatrix}
		\lambda_1 v_1 \\
		\lambda_2 v_2 \\
		\vdots \\
		\lambda_n v_n
	\end{pmatrix},
\]
where some of $\lambda_i$ might be zero.
Without loss of generality, suppose that $\lambda_1, \dots, \lambda_k \neq 0$ while $\lambda_{k+1}, \dots, \lambda_n = 0$.
Image of any nonzero vector in the span of $\set{v_1, \dots, v_k}$ under $T$ is nonzero, while image of any vector in the span of $\set{v_{k+1}, \dots, v_n}$ under $T$ is the zero vector, thus:
\begin{align*}
	\im T & = \spn \set{v_1, \dots, v_k}, \\
	\ker T & = \spn \set{v_{k+1}, \dots, v_n}.
\end{align*}
Since vectors of $B$ are orthogonal, thus $\im T$ and $\ker T$ are orthogonal subspaces that together span $V$.
We conclude that $(\im T) \oplus (\ker T) = V$, as required.

\end{proof}

\begin{tcolorbox}
b) Prove that $T$ is an orthogonal projection onto $\im T$ if and only if, and addition to being symmetric, $A^2=A$.
\end{tcolorbox}

\begin{proof}

Operator $T$ is orthogonal if and only if $T$ is a projection and $\im T$ and $\ker T$ are orthogonal.
By definition, $T$ is a projection if and only if $T^2 = T$.
Let $A$ be the matrix of $T$ with respect to an arbitrary basis.

($\Rightarrow$)
Suppose, $A^2 = A$ and $A$ is symmetric.
By definition, $A$ is a projection matrix, thus $T$ is a projection.
As we have seen in a), since $A$ is symmetric, $\im T$ and $\ker T$ are orthogonal.
We conclude that $T$ is an orthogonal projection.

($\Leftarrow$)
Suppose that $T$ is an orthogonal projection, which means that $T$ is a projection and $\im T$ is orthogonal to $\ker T$.
It immediately follows that $A^2=A$.
We will prove that matrix $A$ is symmetric, i.e. $A^t = A$.

Consider arbitrary vectors $v, w \in V$.
Since $T$ is an orthogonal projection onto $\im T$, we can res present each vector $v$ and $w$ as a sum of two orthogonal vectors:
\[
	v = v_n + \pi(v), \>\> w = w_n + \pi(w).
\]
From this we have:
\[
	\bilform{Tv, w} 
	= \bilform{\pi(v), w_n + \pi(w)} 
	= \bilform{\pi(v), w_n} + \bilform{\pi(v), \pi(w)}
	= \bilform{\pi(v), \pi(w)}
\]
since $\bilform{\pi(v), w_n} = 0$.
Conversely: 
\[
	\bilform{v, Tw} 
	= \bilform{v_n + \pi(v), \pi(w)} 
	= \bilform{v_n, \pi(w)} + \bilform{\pi(v), \pi(w)}
	= \bilform{\pi(v), \pi(w)}.
\]
We have that $\bilform{Tv, w} = \bilform{v, Tw}$ for arbitrary vectors $v,w \in V$, which means that $T$ is self-adjoint.
Matrix of a real self-adjoint operator is symmetric, as required.

\end{proof}


\end{document}
