\documentclass{article}
\usepackage[utf8]{inputenc}
\usepackage[english]{babel}
\usepackage{amsmath}
\usepackage[]{amsthm}
\usepackage[]{amssymb} 
\usepackage{mathrsfs}
\usepackage{tcolorbox}
\usepackage{nicefrac}
\usepackage{mathtools}

\theoremstyle{definition}
\newtheorem*{claim}{Claim}
\newtheorem*{corollary}{Corollary}
\DeclareMathOperator{\adj}{\operatorname{adj}}
\DeclareMathOperator{\im}{\operatorname{im}}
\newcommand{\R}{\mathbb{R}}
\newcommand{\Z}{\mathbb{Z}}
\newcommand{\N}{\mathbb{N}}
\newcommand{\F}{\mathbb{F}}
\newcommand{\GL}{\operatorname{GL}}
\newcommand{\SL}{\operatorname{SL}}
\newcommand{\GLnR}{\GL_n(\R)}
\newcommand{\SLnR}{\SL_n(\R)}
\newcommand{\sgn}{\operatorname{sgn}}
\DeclarePairedDelimiter\floor{\lfloor}{\rfloor}
\DeclarePairedDelimiter\set{\{}{\}}
\DeclarePairedDelimiter\abs{\lvert}{\rvert}
\newcommand{\restrict}[1]{ \big|_{#1} }



\title{18.701: Problem Set 3}
\author{Dmitry Kaysin}
\date{March 2020}
\begin{document}
\maketitle 


\subsection*{Problem 1}

\begin{tcolorbox}
Let $\F_p$ be a prime field, and let $V = \F_p^2$.
\\

a) Prove that the number of bases of $V$ is equal to the order of the general linear group $\GL_2(\F_p)$.
\end{tcolorbox}

\begin{proof}

$B = \set{(1,0), (0,1)}$ is a basis of $V$, thus $\dim V = 2$.
By Artin 3.5.9, any basis $B'$ of $V$ can be represented as $B' = BP$, where $P$ is a unique invertible $2 \times 2$ matrix with entries in $\F_p$.
Such matrices form the general linear group $\GL_2(\F_p)$.
Therefore, the number of unique bases of $V$ is equal to the order of $\GL_2(\F_p)$.

\end{proof}

\begin{tcolorbox}
b) Prove that the order of the general linear group $\GL_2(\F_p)$ is 
\[ p(p+1)(p-1)^2, \]
and the order of the special linear group $\SL_2(\F_p)$ is
\[ p(p+1)(p-1). \]
\end{tcolorbox}

\begin{proof}
Order of $\GL_2(\F_p)$ is equal to the number of possible $2 \times 2$ matrices with entries in $\F_p$ and a non-zero determinant.
We start from the first row.
There are $p^2$ 2-combinations with repetition from $\F_p = \set{0, 1, \dots, p-1)}$.
Therefore, there are $p^2-1$ choices of how to compose the first row of the matrix (we exclude $(0,0)$).
The second row can have any 2-combination with repetition from $\F_p$ except those that are equal to the first row multiplied by some $c \in \F_p$ (including $0$).
Therefore, there are $p^2 - p$ choices of how to compose the second row. We conclude that there are
\[ (p^2-1)(p^2-p) = (p-1)(p+1)(p-1)p = p(p+1)(p-1)^2 \]
choices of how to compose $2 \times 2$ invertible matrix with entries in $\F_p$, which is the order of $\GL_2(\F_p)$, as required.

Consider determinant homomorphism $\det : \GL_2(\F_p) \to \F_p^\times$, where $\F_p^\times$ is the multiplicative group of field $\F$.
The special linear group $\SL_2(\F_p)$ is the kernel of $\det$. By Lagrange's theorem:
\begin{equation} \label{eq:1}
    \abs{ \GL_2(\F_p) } = \abs{ \ker \det } \cdot \abs{ \im \det }
\end{equation} 

We notice that homomorphism $\det$ is surjective, i.\,e. $\im \det = \F_p^\times$. 
Indeed, we can construct invertible matrix with determinant $a$ for every $a \in \F_p^\times$:
\[
    \set*{
    \begin{pmatrix}
        a & 0 \\
        0 & 1
    \end{pmatrix}
    : a \in F_n^\times 
    }
\]
Therefore, $\abs{\im \det} = \abs{\F_p^\times} = p-1$.
Substituting to (\ref{eq:1}) we have:
\begin{gather*}
    p(p+1)(p-1)^2 = \abs{\ker \det} \cdot (p-1), \\
    \abs{\ker \det} = p(p+1)(p-1).
\end{gather*}
Therefore, order of $\SL_2(\F_p)$ is $p(p+1)(p-1)$, as required.

\end{proof}


\subsection*{Problem 2}

\begin{tcolorbox}
Let $\GL$ denote the group $\GL_2(\F_3)$ of invertible matrices with entries modulo $3$.
This group operates on 2-dimensional vectors with entries mod $3$ by matrix multiplication, as usual.

There are $9$ vectors modulo $3$, and four pairs $\pm v$ of nonzero vectors, namely
\[
    s_1 =
    \pm \begin{pmatrix}
        1 \\
        0
    \end{pmatrix}, \quad
    s_2 =
    \pm \begin{pmatrix}
        0 \\
        1
    \end{pmatrix}, \quad
    s_3 =
    \pm \begin{pmatrix}
        1 \\
        1
    \end{pmatrix}, \quad
    s_4 =
    \pm \begin{pmatrix}
        1 \\
        -1
    \end{pmatrix}.
\]
The elements of $\GL$ permute the nonzero vectors, and they also permute the pairs of nonzero vectors.
Sending a matrix to the permutation it defines gives us a homomorphism $\varphi$ from $\GL$ to the symmetric group $S_4$ of permutations of $\set{s_1, s_2, s_3, s_4}$.
For example, if
$E =
    \bigl( \begin{smallmatrix}
    1 & 1 \\
      & 1
    \end{smallmatrix} \bigr)
$,
then $\varphi(E)$ is the 3-cycle $(s_2, s_3, s_4)$.
\\

a) Show that $\varphi$ is a surjective map, and determine its kernel.
\end{tcolorbox}

\begin{proof}

We first find the kernel of $\varphi$.
We compose $2 \times 4$ matrix from column-vectors that represent $s_1, s_2, s_3$ and $s_4$:
\[  S =
    \begin{bmatrix}
        \pm \begin{pmatrix}
            1 \\ 0
        \end{pmatrix} \>\>
        \pm \begin{pmatrix}
            0 \\ 1
        \end{pmatrix} \>\>
        \pm \begin{pmatrix}
            1 \\ 1
        \end{pmatrix} \>\>
        \pm \begin{pmatrix}
            1 \\ -1
        \end{pmatrix}
    \end{bmatrix}
\]
Matrix 
$
    A =
    \begin{bmatrix}
    a & b \\
    c & d
    \end{bmatrix}
$
such that $A \in \ker \varphi$ must preserve vectors $(s_1,s_2,s_3,s_4)$, thus:
\[ AS = S, \]
up to a sign of each vector. Multiplying:
\begin{multline*}
    \begin{bmatrix}
        \begin{pmatrix}
            \pm a \\ 0
        \end{pmatrix} \>\>
        \begin{pmatrix}
            0 \\ \pm d
        \end{pmatrix} \>\>
        \begin{pmatrix}
            (\pm a) + (\pm b) \\  (\pm c) + (\pm d)
        \end{pmatrix} \>\>
        \pm \begin{pmatrix}
            (\pm a) + (\pm b) \\  (\pm c) - (\pm d)
        \end{pmatrix}
    \end{bmatrix}
    = \\ =
    \begin{bmatrix}
        \pm \begin{pmatrix}
            1 \\ 0
        \end{pmatrix} \>\>
        \pm \begin{pmatrix}
            0 \\ 1
        \end{pmatrix} \>\>
        \pm \begin{pmatrix}
            1 \\ 1
        \end{pmatrix} \>\>
        \pm \begin{pmatrix}
            1 \\ -1
        \end{pmatrix}
    \end{bmatrix}
\end{multline*}
From the first and the second vectors we can see that $a = \pm 1, \> b = 0, \> c = 0, \> d = \pm 1$. From the third vector we can see that $a = d$. We conclude that elements
\[
    I_\pm = 
    \pm \begin{bmatrix}
    1 & 0 \\
    0 & 1
    \end{bmatrix}
\]
form the kernel of $\varphi$.
As per Lagrange's theorem:
\[ \abs{\GL} = \abs{\ker \varphi} \cdot \abs{\im \varphi} \]
Using the result of Problem 1, the order of $\GL$ is $48$, thus
\begin{gather*}
    48 = 2 \cdot \abs{\im \varphi} \\    
    \abs{\im \varphi} = 24
\end{gather*}
The order of symmetric group $S_4$ is $4! = 24$, therefore $\abs{\im \varphi} = \abs{S_4}$ and $\varphi$ must be surjective.

\end{proof}

\begin{tcolorbox}
b) Determine the subgroup of $\GL$ that corresponds, by the Correspondence Theorem, to the alternating subgroup $A_4$ of $S_4$.
\end{tcolorbox}

The special linear group $\SL_2(\F_3)$ corresponds to $A_4$.

\begin{proof}
Denote $\SL = \SL_2(\F_3)$.
Using the result of Problem 1, the order of $\SL_2(\F_3)$ is $24$.
By the Correspondence Theorem, since $\varphi$ is onto $S_4$ and $\ker \varphi \leq \SL$, there exists a subgroup $G$ of $S_4$ such that $G = \varphi(\SL)$ and
\begin{gather*}
    \abs{ \SL } = \abs{ \ker \varphi } \cdot \abs{ \varphi(\SL) } \\
    24 = 2 \cdot \abs{ \varphi(\SL) } \\
    \abs{ \varphi(\SL) } = \abs{ G } = 12.
\end{gather*}

We remember that the special linear group $\SL$ is generated by elementary row-addition matrices of the form:
\[
    \begin{pmatrix}
        1 & a \\
        0 & 1
    \end{pmatrix},
    \quad
    \begin{pmatrix}
        1 & 0 \\
        a & 1
    \end{pmatrix},
    \quad a \in \F_3, \> a \neq 0.
\]
We check images of elementary matrices under $\varphi$:
\begin{gather*}
    \varphi
    \begin{pmatrix}
        1 & 1 \\
        0 & 1
    \end{pmatrix} = 
    (s_2 s_3 s_4),
    \quad
    \varphi
    \begin{pmatrix}
        1 & -1 \\
        0 & 1
    \end{pmatrix} = 
    (s_2 s_4 s_3),
    \\
    \varphi
    \begin{pmatrix}
        1 & 0 \\
        1 & 1
    \end{pmatrix} = 
    (s_1 s_3 s_4),
    \quad
    \varphi
    \begin{pmatrix}
        1 & 0 \\
        -1 & 1
    \end{pmatrix} = 
    (s_1 s_4 s_3).
\end{gather*}
We can see that images of the generators of $\SL$ are 3-cycles, even permutations, each distinct element of $A_4$ of order $3$.

Consider arbitrary element $B \in \SL$, which can be represented as a product of elementary matrices:
\[ B = E_1 \cdot E_2 \cdots E_n \]
Consider its image under $\varphi$:
\begin{gather*}
    \varphi(B) = \varphi(E_1 \cdot E_2 \cdots E_n) \\
    \varphi(B) = \varphi(E_1) \cdot \varphi(E_2) \cdots \varphi(E_n)
\end{gather*}
Each $\varphi(E_k)$ above is a 3-cycle.
Product of arbitrary number of 3-cycles is an even permutation, thus an element of $A_4$.
Therefore, $\varphi(\SL) \leq A_4$. Since $\abs{\varphi(\SL)} = \abs{A_4}$, we conclude that $\varphi(\SL) = A_4$ and $\SL$ corresponds to $A_4$.

\end{proof}

\begin{tcolorbox}
c) Determine the subgroup of $S_4$ that corresponds to the subgroup of $\GL$ of upper triangular matrices.
\end{tcolorbox}

\begin{proof}

Denote $U \leq \GL$, the subgroup of upper triangular matrices in $\GL$:
\[
    \begin{pmatrix}
        a & b \\
        0 & c
    \end{pmatrix}
\]
We first find the order of $U$. Since $U$ must be invertible, $a \neq 0$ and $c \neq 0$. Therefore, there are $2$ choices of $a$, $2$ choices of $c$ and $3$ choices of $b$, for the total of $12$ possible matrices in $U$. Therefore, $\abs{U}=12$.

We notice that $\ker \varphi \leq U$, therefore the Correspondence Theorem applies, and there must exist a subgroup $\varphi(U) \leq S_4$ of order $6$.

There are $4$ subgroups of order $6$ in $S_4$, specifically groups of permutations of $\set{s_1,s_2,s_3}, \set{s_2,s_3,s_4}, \set{s_1,s_2,s_4}$ and $\set{s_1,s_3,s_4}$, each of which is isomorphic to the symmetric group $S_3$.

We check image of one of the elements of $U$ under $\varphi$:
\[
    \varphi
    \begin{pmatrix}
        1 & 1 \\
        0 & 1
    \end{pmatrix} = 
    (s_2 s_3 s_4),
\]
which is an element of the group of permutations of $\set{s_2,s_3,s_4}$.
Therefore, $U$ corresponds to the group of permutations of $\set{s_2,s_3,s_4}$, which is isomorphic to $S_3$.

\end{proof}


\subsection*{Problem 3}

\begin{tcolorbox}
Let $a = (a_1, \dots a_k)$ and $b = (b_1, \dots b_k)$ be points in $k$-dimensional space $\R^k$.
A path from $a$ to $b$ is a continuous function on the unit interval $[0,1]$ with values in $\R^k$, a function $X: [0,1] \to \R^k$, sending $t \mapsto X(t) = (x_1(t), \dots x_k(t))$, such that $X(0) = a$ and $X(1) = b$.
If $S$ is a subset of $\R^k$ and if $a$ and $b$ are in $S$, define $a \sim b$ if $a$ and $b$ can be joined by a path lying entirely in $S$.

a) Show that $\sim$ is an equivalence relation on $S$. Be careful to check that any paths you construct stay within the set $S$.
\end{tcolorbox}
\begin{proof}

We first check reflexivity.
Consider $X: t \mapsto a$, constant function.
$X(t)$ is continuous; $X([0,1]) \subseteq S$ since $a \in S$; and $X(0) = X(1) = a$.
Therefore, $a \sim a$.

We then check symmetry.
Consider points $a,b \in S$ such that $a \sim b$.
There must exist continuous function $X(t)$ such that $X[0,1] \subseteq S$ and $X(0) = a, X(1) = b$.
Consider function $Y(t) = X(1-t)$.
We notice that $Y(t)$ is a composition of continuous functions $f(t) = (1-t)$ and $X(t)$, thus it must be continuous; $Y[0,1] = X[0,1] \subseteq S$; $Y(0) = X(1) = b, \> Y(1) = X(0) = a$. Therefore, $b \sim a$.

Lastly, we check transitivity.
Consider points $a,b,c \in S$ such that $a \sim b, \> b \sim c$.
There must exist continuous function $X(t)$ such that $X[0,1] \subseteq S$ and $X(0) = a, X(1) = b$ and function $Y(t)$ such that $Y[0,1] \subseteq S$ and $Y(0) = b, Y(1) = c$. Consider function $Z(t)$:
\[
    Z(t) =
    \begin{cases}
        X(2t), & \text{ if $t \in [0, 0.5]$}, \\
        Y(2t-1), & \text{ if $t \in (0.5, 1]$}.
    \end{cases}
\]
This is a piecewise function that is continuous since both $X$ and $Y$ are continuous on $[0,1]$ and $X(1) = Y(0)$.
We also notice that $Z[0,1] = X[0,1] \cup Y(0,1] \subseteq S$ and $Z(0) = X(0) = a, \> Z(1) = Y(1) = c$.
Therefore, $a \sim c$.

We conclude that $\sim$ is an equivalence relation.

\end{proof}

\begin{tcolorbox}
b) A subset $S$ is path connected if $a \sim b$ for any two points $a$ and $b$ in $S$.
Show that every subset $S$ is partitioned into path-connected subsets with the property that two points in different subsets cannot be connected by a path in $S$.
\end{tcolorbox}

\begin{proof}

Equivalence relation $\sim$ on $S$ induces a partition on $S$.
Sets in a partition represent different equivalence classes and are disjoint.
This means that every point of $S$ belongs to one and only one of the partition sets and for any $a,b \in S$ that belong to different sets of the partition, $a \nsim b$.
From the definition of path-connectedness, such points cannot be connected by a path in $S$.

\end{proof}


\subsection*{Problem 4}

\begin{tcolorbox}
The set of $n \times n$ matrices can be identified with the space $\R^{n \times n}$. Let $G$ be a subgroup of $\GLnR$. With the notation of the previous exercise, prove:

a) If $A,B,C$ and $D$ are in $G$, and if there are paths in $G$ from $A$ to $B$ and from $C$ to $D$, then there is a path in $G$ from $AC$ to $BD$.
\end{tcolorbox}

\begin{proof}

We say that function $\varphi : [0,1] \to \GLnR$ is continuous if $\varphi$ is continuous in each of its entries.
Since $A \sim B$, there exists function $X:[0,1] \to G$ such that $X(0) = A, \> X(1) = B$ and $\rho \circ X$ is continuous on $[0,1]$.
Since $C \sim D$, there exists function $Y:[0,1] \to G$ such that $Y(0) = C, \> Y(1) = D$ and $\rho \circ Y$ is continuous on $[0,1]$.

Since $G$ is a subgroup, $AC, AD,$ and $BD$ must be elements of $G$.

We first prove that $AC \sim AD$ with a path $f : t \mapsto A Y(t)$.
We notice that $f(0) = A Y(0) = AC$ and $f(1) = A Y(1) = AD$ (endpoints match).
We know that $Y$ is continuous on $[0,1]$.
Left-multiplication by $A$ is a linear map, thus continuous function.
Composition of continuous functions is continuous, thus $f$ is continuous on $[0,1]$.
We also notice that for any $t \in [0,1]$ : $Y(t) \in G$.
Then $A Y(t) \in G$.
Therefore, $f[0,1] = A \cdot Y[0,1] \subseteq G$ (path is in $G$).
We conclude that $AC \sim AD$.

We then prove that $AD \sim BD$ with a path $g : t \mapsto X(t) D$.
We notice that $g(0) = X(0) D = AD$ and $g(1) = X(1) D = BD$ (endpoints match).
We know that $X$ is continuous on $[0,1]$.
Right-multiplication by $D$ is a linear map, thus continuous function.
Composition of continuous functions is continuous, thus $g$ is continuous on $[0,1]$.
We also can see that $g[0,1] = X[0,1] \cdot D \subseteq G$ via the same reasoning as above (path is in $G$).
We conclude that $AD \sim BD$.

Since $AC \sim AD$ and $AD \sim BD$, by transitivity, $AC \sim BD$.

\end{proof}

\begin{tcolorbox}
b) The set of matrices that can be joined to the identity $I$ forms a normal subgroup of $G$. (It is called the connected component of $G$).
\end{tcolorbox}

\begin{proof}

Consider set $H \subseteq \GLnR$ such that for all $A \in H : A \sim I$, where $I$ is the identity element of $\GLnR$.
We note that $I \sim I$ by reflexivity.
Thus, $H$ contains the identity element.

For arbitrary $A \in H$
\begin{gather*}
    I \sim A, \\
    \intertext{then, by the result of part a):}
    A^{-1} \sim A A^{-1} = I.
\end{gather*}
Thus, $H$ contains inverses.

For arbitrary $A,B \in H$:
\begin{gather*}
    A \sim I \sim B \sim B^{-1}, \\
    \intertext{then, by the result of part a):}
    AB \sim B^{-1} B, \quad \quad AB \sim I.
\end{gather*}
Thus, $H$ is closed under matrix multiplication.

Therefore, $H$ is a subgroup of $\GLnR$.

We will now prove that $H$ is normal. Consider arbitrary $A \in H$ and $C \in \GLnR$.
\begin{gather*}
    A \sim I, \\
    CA \sim CI, \\
    CAC^{-1} \sim CIC^{-1}, \\
    CAC^{-1} \sim I, \\
    CAC^{-1} \in H.
\end{gather*}
Thus, $H$ is normal.

\end{proof}


\subsection*{Problem 5}

\begin{tcolorbox}
a) The group $\SLnR$ is generated by elementary matrices of the first type (row addition).
Use this fact to prove that $\SLnR$ is path-connected.
\end{tcolorbox}

\begin{proof}

Any elementary matrix
\[
    E_a = \begin{pmatrix}
        1 & a \\
        0 & 1
    \end{pmatrix},
    \quad \text{ for $a \in \R$}
\]
can be joined to $I$, the identity matrix, via the path
\[
    \varphi : \> 
    t \longmapsto
    I + \begin{pmatrix}
    0 & at \\
    0 & 0
    \end{pmatrix}
\]
Thus, $E_a \sim I$. The same is true for elementary matrices $E_a^T$.

Any matrix $A \in \SLnR$ can be represented as a product of elementary matrices $E_a$ and $E_a^T$:
\[ A = E_k \cdots E_2 E_1 I. \]
Using the result of Problem 4 a):
\begin{align*}
    I & \sim I, \\
    E_1 \cdot I & \sim I \cdot I, \\
    E_2 \cdot E_1 \cdot I & \sim I \cdot I \cdot I, \\
    & \cdots \\
    A & \sim I.
\end{align*}
Thus, $\SLnR$ is path-connected.

\end{proof}

\begin{tcolorbox}
Show that $\GLnR$ is a union of two path-connected subsets and describe them.
\end{tcolorbox}
\begin{proof}

In this exercise we will use the group notation with lowercase letters representing elements of the group, i.e. matrices; $e$ being the identity element of $\GLnR$; and uppercase letters representing subgroups of $\GLnR$.

Consider homomorphism $\varphi = \sgn \circ \det$, where $(\sgn)$ is a sign function and $(\det)$ is a determinant homomorphism.
One can easily see that $\varphi$ is indeed a homomorphism that sends $\GLnR$ to $\Z_2$.

Denote $H^+$, the set $\set{ x \in \GLnR : \varphi(x) = 1}$ (matrices with positive determinant).
Denote $H^-$, the set $\set{ x \in \GLnR : \varphi(x) = -1}$ (matrices with negative determinant).
$H^+$ is a kernel of $\varphi$.
Therefore, by the first isomorphism theorem, $\GLnR \setminus H^+ \cong \Z_2$.
Thus, there are only two elements in $\GLnR \setminus H^+$, specifically $H^+$ and its coset $H^-$.

We will first show that $H^+$ is path-connected.
As we have seen in the Problem 5 a), $\SLnR$ is path-connected.
To prove $H^+$ is path-connected it suffices to show that any matrix $a \in H^+$ can be path-connected to some matrix $c \in \SLnR$.
Consider function
\[
f : t \longmapsto
\begin{pmatrix}
    (1-t) + t\dfrac{1}{\det a} & 0 \\
    0 & 1
\end{pmatrix}
\]
We claim that function $f(t)a$ for $t \in [0,1]$ is the required path.
Indeed
\[ f(0)a = ea = a, \quad \det f(1)a = \det \left( \nicefrac{1}{\det a} \right) \cdot \det a = 1, \]
thus $f(1)a = c$, where $c$ is some element of $\SLnR$.
Every $f(t)$ has positive determinant, thus every $f(t)a$ is in $H^+$. 
Finally, $f(t)$ is continuous and right-multiplication is continuous, thus $f(t)a$ is continuous.
Therefore, $a \sim c \sim e$ and we conclude that $H^+$ is path-connected.

We will now show that $H^-$ is path-connected.
Consider arbitrary $b \in H^-$.
Since $H^-$ is a coset of $H^+$, there exists $a \in H^+$ such that $b = xa$ for some $x \in H^-$.
We know that $a \sim e$.
By the result of Problem 4 a):
\[ xa \sim xe, \quad b \sim xe. \]
Since $xe \in H^-$, we conclude that all elements of $H^-$ can be path-connected via $xe$, thus $H^-$ is path-connected. We also note that no two elements $b \in H^-$, $a \in H^+$ are path-connected, otherwise, $b$ would be path-connected to $e$ and by the result of the Problem 4 b), $b$ would be an element of $H^+$, while $H^-$ and $H^+$ are disjoint.

We conclude that both $H^+$ and $H^-$ are path-connected and they together partition $\GLnR$, as requested.

\end{proof}


\end{document}
